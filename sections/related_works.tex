\section{Related Works} \label{sec:related-works}

The detection of money laundering through machine learning has 
evolved over the past decade, with research spanning 
core detection methodologies, human-AI collaboration, 
and advanced neural architectures. This 
section reviews literature within those fields. However, due to the 
specific niche, role, and organizational placement 
of the proposed solution within 
\gls{lsb}, certain aspects of this body of 
work are less directly applicable in our 
context and are discussed inline below.

\subsection{Traditional Machine Learning Approaches for AML}

The application of machine learning to anti-money laundering detection has been 
widely studied. Early reviews by Chen et 
al.~\cite{chen2018learning} 
establish foundational taxonomies of machine learning techniques, including 
anomaly detection, clustering, classification, and network analysis for suspicious 
transaction detection. Jensen and 
Iosifidis~\cite{jensen2023fightingmoneylaunderingstatistics} 
provide a more recent framework, distinguishing between 
client risk profiling and suspicious behaviour flagging, addressing 
class imbalance challenges, feature engineering issues, and evaluation 
metrics appropriate for imbalanced datasets. 

Building on these foundations, recent work has demonstrated the 
practical effectiveness of supervised learning approaches on real-world 
banking data. Jullum et al.~\cite{jullum2020detecting} use 
data from DNB, 
Norway's largest bank, to show that including non-reported alerts in training 
improves model performance. They also introduce performance measures 
for comparing ML approaches to existing rule-based AML systems. 
Namdar et 
al.~\cite{namdar2025antimoneylaunderingmachinelearning} 
present a 16-step 
supervised learning pipeline achieving $0.961$ \gls{aucroc} on 
high-risk client 
identification, with emphasis on explainable AI modules for feature 
importance, which improves understanding of how models prioritize clients 
for investigation.

One recent survey by Fan et 
al.~\cite{fan2025deeplearningapproachesantimoney} reviews 
deep learning 
approaches for mobile transaction AML, covering \glspl{cnn}, 
\glspl{rnn}, transformers, 
and \glspl{gnn} while addressing challenges of false 
positives, 
limited data availability, and real-time 
detection requirements.\\
Together, the above works provide a 
general theoretical basis, and serve as a preliminary guide for our model 
selection.

\subsection{Human-AI Collaboration and Modelling Analyst Behaviour}

An underexplored area is how human analysts and AI systems collaborate 
in AML detection, and how analyst decision-making patterns can be 
captured and replicated. Recent 
research~\cite{kadam2024enhancingfinancialfrauddetection} 
demonstrates how Subject Matter Expert (SME) feedback annotations can be propagated 
through transaction graphs to improve fraud detection models, showing 
$7.24\%$ average improvement in AUC when human expertise from AML analysts 
is incorporated into ML systems through feedback loops.

Nagarakanti~\cite{nagarakanti2025human} provides analysis 
showing that experienced financial professionals improve AI-generated risk 
assessments by $42.8\%$ when market conditions deviate from historical 
patterns, 
and that human interventions improve lending outcomes by $19.7\%$ for edge 
cases. 
The documented collaborative fraud detection workflow, in which analysts 
correctly resolve $81.7\%$ of complex anomaly cases, illustrates the importance 
and impact of expert analyst decision-making within human–AI systems.

A methodological framework for evaluating Human-AI 
Collaboration 
systems~\cite{fragiadakis2025evaluatinghumanaicollaborationreview} includes 
fraud detection as a domain of study, examining task allocation, 
interaction patterns, and how humans and AI contribute to collaborative outcomes. 
The framework discusses detection rates (true/false positives) as evaluation 
metrics and provides structured approaches for evaluating collaboration modes 
(AI-Centric, Human-Centric, and Symbiotic), which is directly applicable to 
understanding how AML analysts should interact with and be modelled within AI 
systems.\\
The insights of these papers align closely with the objectives of the proposed 
system, which only aims to assist \gls{aml} employees, not replace them. 
Particularly, we explore the possibility that employee-based ensembles may 
exhibit particular aptitude for this task. The results will be discussed in 
\autoref{sec:results}.

\subsection{Addressing Class Imbalance in AML Detection}

Class imbalance is a practical challenge in AML detection, 
as fraudulent transactions often constitute less than $1\%$ of 
all data. Liu et al.~\cite{liu2021pick} address this 
challenge 
through PC-GNN, a graph-based approach using label-balanced sampling to 
construct sub-graphs for mini-batch training, neighbourhood sampling to 
choose relevant neighbours, and aggregation across different relations. 
Tests on YelpChi and Amazon benchmark datasets show improvements 
over state-of-the-art baselines.

Ruchay et al.~\cite{ruchay2023imbalanced} focus on handling 
class imbalance in fraudulent bank transactions using the 
CreditCardFraud\footnote{https://www.kaggle.com/datasets/mlg-ulb/creditcardfraud}
 dataset. Their work employs Tomek Links' resampling algorithm and emphasizes 
appropriate evaluation metrics for imbalanced data (Precision, Recall, F1, IBA) 
rather than accuracy alone, achieving $99.99\%$ accuracy with proper handling 
of imbalance.\\

